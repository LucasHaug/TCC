\def \MOISEp {$\mathcal{M}OISE^+$} 

No mundo acadêmico, as competições universitárias de robótica têm um grande papel no desenvolvimento do cenário da área tanto no quadro nacional quanto no internacional, incentivando a pesquisa e a concepção de sistemas robóticos. No contexto dessas competições, uma das categorias de robôs mais desafiadora e estimulante é a \textit{IEEE Very Small Size Soccer (VSSS)}, que visa desenvolver uma solução completa de engenharia para um time de robôs que jogam futebol autonomamente. A Equipe ThundeRatz de Robótica da Escola Politécnica da USP possui atualmente uma solução para o problema proposto pela categoria, a qual utiliza o \textit{Robotic Operating System (ROS)} para estruturação do sistema, árvores de comportamento para modelagem dos agentes inteligentes que jogam futebol e uma Máquina de Estados Finita para coordenar esses agentes. Essa solução é conhecida como o time ThunderVolt e é o alvo deste trabalho, o qual tem como objetivo aprimorar o time ao desenvolver uma uma nova estratégia de coordenação usando árvores de comportamento, tendo como base um modelo da organização do time feito usando \MOISEp. Além disso, o desempenho a melhoria foi comparado ao sistema anterior usando uma Máquina de Estados Finita e em uma competição acadêmica de robótica, os resultados de ambos os testes serão discutidos.