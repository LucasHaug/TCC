\def \MOISEp {$\mathcal{M}OISE^+$} 
\def \MOISEpBf {$\mathbf{\mathcal{M}OISE^+}$} 

\chapter{Background}
\label{ch:background}

This chapter explains the theoretical foundation for understanding this work, from the basics of multi-agent systems and modeling these systems to different control architectures used in robotics.

\section{Multi-agents systems}

a

\section{\MOISEpBf}

% exemplos

\section{Control Architectures}

%  - AC multiplos robos, ambientes dniamicos

% https://arxiv.org/pdf/1709.00084.pdf
% comparacao entre arquiteturas

\section{Finite State Machines}

\section{Hierarchical State Machines}

\section{Behavior Trees}

% https://www.overleaf.com/learn/latex/Algorithms

% arrumar ref na implementacao sobre reactive

% \linkhttps{https://en.wikipedia.org/wiki/Behavior_tree_(artificial_intelligence,_robotics_and_control)}
% https://www.overleaf.com/learn/latex/TikZ_package
% https://forum.thunderatz.org/t/arquiteturas-de-controle/3640
% https://www.behaviortree.dev/docs/learn-the-basics/BT_basics

\subsection{Types of Nodes}

\subsubsection{Root Node} 

\subsubsection{Control Nodes} 
\label{subsubsec:control_nodes}

\subsubsection{Action Nodes}

\subsubsection{Condition Nodes}

\subsubsection{Decorator Nodes}

\subsubsection{Subtree Nodes}

\subsection{Reactivity}

\subsection{Blackboard}


\cite{BlackboardDesignPattern}


\section{BT vs other Control Architectures}
