\chapter{Results}
\label{ch:results}

\section{Comparison between models}

This differs from the previous system with the FSM, where the initialization process occurred outside the FSM during the definition of the FSM's entry state.

flags use penalty mode, use two strikers

These configurations were modeled in the FSM as entry states, however, in the behavior tree they were adapted as flags, being able to be activated and deactivated.

\section{Tests and Evaluation}

% numero de testes
% setup de testes
% statisticas
% jogos

\section{Discussion}

% ṕerformance pelo menos igual

In order to compare the performance of the system that uses a finite state machine with the system that uses a behavior tree to model the organization, all other parts of the system must be kept unchanged. This is necessary so that test games can be made, where the two systems with different strategies would face each other, so that some game metrics can be collected for statistical comparison of the solutions, comparing metrics such as ball possession, number of goals, time in each quadrant of the field, among others.

falar sobre os requisitos
