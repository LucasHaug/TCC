\chapter{Methodology}
\label{ch:methodology}

An analysis of the feasibility of adopting the MOISE \cite{MOISEp} model was carried out to define the new strategy, based on the modeling of the current team using this model, which will be presented in Chapter \ref{ch:target_system}. From this mapping, it was observed that \textit{Coach} implicitly had an organizational model for the team. In the current solution \textit{Coach} is described as a Finite State Machine (FSM).

Thus, for the development of the new strategy, the team organization will be modeled using the \textit{Moise} \cite{MOISEp} model. The modeling will be specified through a behavior tree (BT), in order to enable the implementation of the team organization in a decentralized way. In addition, a package will be developed to collect system metrics, so that the improvements obtained can be evaluated.

After the system improvements are implemented, all changes will be tested in academic competitions. The initial intention is to make preliminary tests in the \textit{Competição Brasileira de Futebol de Robôs Simulado} (CBFRS), a competition in the style of a round-robin tournament, in which the team is already participating. After verifying the team's performance in this competition, improvements will be introduced to participate in a second competition, the \textit{IRONCup 2023} \cite{IRONCup2023}.

\section{Activities and Schedule}

To this end, the following activities will be carried out:

\begin{itemize}
    \item {\bf At. 1} Writing of the monograph;
    \item {\bf At. 2} Modeling of the Coach's BT;
    \item {\bf At. 3} Review of the Coach's BT;
    \item {\bf At. 4} Integration of the Coach's package to use BTs;
    \item {\bf At. 5} Development of the nodes of the BT;
    \item {\bf At. 6} Assembly of the Coach's BT;
    \item {\bf At. 7} Coach's migration from the FSM to the BT;
    \item {\bf At. 8} Tests of the system with the FSM against the system with the BT;
    \item {\bf At. 9} Creation of the metrics analysis package;
    \item {\bf At. 10} Definition of metrics for performance analysis;
    \item {\bf At. 11} Creation of a library to perform analyses;
    \item {\bf At. 12} Creation of a ROS node to collect metrics.
\end{itemize}

Distributing these activities over the weeks, the following distribution of activities is obtained.

\begin{table}[!htbp]
    \centering
    \resizebox{\textwidth}{!}{%
        \begin{tabular}{|c|c|c|c|c|c|c|c|c|c|c|c|c|}
        \hline
                    & At. 1 & At. 2 & At. 3 & At. 4 & At. 5 & At. 6 & At. 7 & At. 8 & At. 9 & At. 10 & At. 11 & At. 12 \\ \hline
        8/5 a 14/5  & X     & X     &       &       &       &       &       &       &       &        &        &        \\ \hline
        15/5 a 21/5 &       & X     &       &       &       &       &       &       &       &        &        &        \\ \hline
        22/5 a 28/5 & X     & X     &       &       &       &       &       &       &       &        &        &        \\ \hline
        29/5 a 4/6  & X     & X     & X     &       &       &       &       &       &       &        &        &        \\ \hline
        5/6 a 11/6  & X     &       & X     &       &       &       &       &       &       &        &        &        \\ \hline
        12/6 a 18/6 &       &       &       &       &       &       &       &       &       &        &        &        \\ \hline
        19/6 a 25/6 & X     &       &       &       &       &       &       &       &       &        &        &        \\ \hline
        26/6 a 2/7  & X     &       &       &       &       &       &       &       &       &        &        &        \\ \hline
        3/7 a 9/7   & X     &       &       &       &       &       &       &       &       &        &        &        \\ \hline
        10/7 a 16/7 &       &       &       & X     & X     &       &       &       &       &        &        &        \\ \hline
        17/7 a 23/7 &       &       &       & X     & X     &       &       &       &       &        &        &        \\ \hline
        24/7 a 30/7 &       &       &       &       &       &       &       &       &       &        &        &        \\ \hline
        31/7 a 6/8  &       &       &       &       & X     &       &       &       & X     & X      &        &        \\ \hline
        7/8 a 13/8  &       &       &       &       & X     &       &       &       &       &        & X      &        \\ \hline
        14/8 a 20/8 &       &       &       &       &       & X     & X     & X     &       &        & X      & X      \\ \hline
        21/8 a 27/8 &       &       &       &       &       &       &       & X     &       &        & X      & X      \\ \hline
        \end{tabular}%
    }
    \caption{Distribution of tasks over the weeks}
\end{table}
