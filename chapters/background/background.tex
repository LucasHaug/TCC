\def \MOISEp {$\mathcal{M}OISE^+$}
\def \MOISEpBf {$\mathbf{\mathcal{M}OISE^+}$}

\chapter{Background}
\label{ch:background}

This chapter will explain the theoretical foundation for understanding this work, from the basics of multi-agent systems and modeling these systems to different control architectures used in robotics.

\section{Multi-agents systems}



\section{\MOISEpBf}

\MOISEp \cite{MOISEp} is a framework for modeling MAS developed by researchers from the University of São Paulo together with researchers from the Ecole Nationale Supérieure des Mines de Saint-Etienne, which extends its predecessor, called MOISE (Model of Organization for multI-agent SystEms) \cite{Moise}. Therefore, to better understand \MOISEp it is first necessary to comprehend the basics concepts of the MOISE model.

MOISE is a framework for designing and developing complex and dynamic organizations for multi-agent systems using an organization-centric point of view. The model provides a structured way of defining an organization so that all the agents work together in a coordinated and efficient manner, this structure is obtained by linking the roles that an agent can play to the plans that need to be executed for the organization to work as a whole. The model is divided into three levels:

\begin{itemize}
    \item \textit{Individual level}: The behavior that needs to be performed for a specific role.
    \item \textit{Social level}: The relationships between the roles.
    \item \textit{Collective level}: The aggregation of roles in large structures.
\end{itemize}

Nevertheless, certain limitations within MOISE required addressing to enhance the framework, such as the lack of the ability to explicitly define global plans for the system within the model. And it was for this reason that the model needed to be extended to the \MOISEp model \cite{MOISEp}. \MOISEp introduces three types of specifications that collectively define the framework: Structural Specification, Functional Specification, and Deontic Specification. These three specifications are essential for defining how the MAS organization works so it can reach its goal.

The \textbf{Structural Specification} delineates the organizational structure, defining the roles that the agents can play, their interrelationships, and group affiliations. This specification allows for the definition of hierarchical relationships, compatibility between roles, role authority, and other relevant characteristics. It also enables the specification of whether one role holds authority over another, the compatibility constraints between roles, and the number of agents that can be assigned to a particular role, among other structural details. An example of a \textbf{Structural Specification} from a MAS can be seen in Figure \ref{fig:moise_ss}.

\begin{figure}
    \centering
    \includegraphics[width=0.75\linewidth]{chapters/background/images/MOISE - SS.png}
    \caption{Structural Specification of a soccer team. Taken from \cite{MOISEp}}
    \label{fig:moise_ss}
\end{figure}

From this example, it is possible to understand the different categories of links between the roles that exist. A link has two attributes, its type, presented in Table \ref{tab:types_of_links_in_moise}, and its scope, which can be intra-group or inter-group. An intra-group link is used to specify that an agent playing the link source role is linked to all agents playing the destination role within the group or any of its sub-groups. Meanwhile, an inter-group link connects despite the groups the agents it connects belongs to.

\def \sourceagent{$a_s$ }
\def \destagent{$a_d$ }

\begin{table}[!htbp]
    \begin{minipage}{\columnwidth}
        \centering
        \begin{tabular}{l l}
            \toprule
            Types         & Meaning                                                        \\
            \midrule
            Acquaintance  & \sourceagent is allowed to have a representation of \destagent \\
            Communication & \sourceagent is allowed to communicate with \destagent         \\
            Authority     & \sourceagent is allowed to have authority over \destagent      \\
            Compatibility & \sourceagent is also allowed to play the destination role      \\
            \bottomrule
        \end{tabular}
        \begin{center}
            \footnotesize
            \emph{Note}: \sourceagent stands for the agent playing the source role of the link \\
            and \destagent stands for the agent playing the destination role. \\
        \end{center}
    \end{minipage}
    \caption{Types of links between the roles in \MOISEp}
    \label{tab:types_of_links_in_moise}
\end{table}

Conversely, the \textbf{Functional Specification} is employed to define a way the goals of the organization can be achieved. It involves defining global plans, a structured way of combining goals, and utilizing a set of global goals to formulate missions. These missions serve as the foundation for the organization's social scheme, and agents within the organization can commit to missions in accordance with the rules defined in the social scheme of how many agents can commit to a specified mission. An example of a social scheme is illustrated in Figure \ref{fig:moise_fs}.

Lastly, the \textbf{Deontic Specification} establishes the relationship between the Structural and Functional Specifications by specifying the permissions and obligations associated with each role in relation to a mission.

\begin{figure}[!h]
    \centering
    \begin{subfigure}{.44\linewidth}
        \centering
        \includegraphics[width=\linewidth]{chapters/background/images/Moise - Social Scheme.png}
    \end{subfigure}
    \hfill
    \begin{subfigure}{.55\linewidth}
        \centering
        \includegraphics[width=0.8\linewidth]{chapters/background/images/Moise - Goals Descriptions.png}
    \end{subfigure}
    \caption{Example of Social Scheme to score a soccer goal. Taken from \cite{MOISEp}}
    \label{fig:moise_fs}
\end{figure}

\section{Control Architectures}

As defined in \cite{BTsInRobotics2}, a control architecture is a way of encoding a robot's functionality, by defining how a specified task is carried out. A control architecture provides a structured form of defining the intelligence of an agent, facilitating its comprehension, development, and debugging. There are many different types of control architectures that have been developed, each featuring a distinct set of tools, rules, and guidelines for organizing how to control a system.

In this section, two of the most common control architectures in robotics are presented, Finite State Machines (FSM) and Hierarchical Finite State Machines (HFSM). In the next section, the control architecture that is the focus of this work is presented, the Behavior Trees (BT).

\subsection{Finite State Machines}

FSMs are a very common mathematical model of computation, a FSM represents a system which, at any moment in time, can only be in one of a finite number of states. A FSM is defined by a list of states, an initial state, and a set of transition functions that determine how the system transits from one state to another, depending on some inputs, in addition to also being able to have final states of the system. The FSM can be represented as a directed graph, where the states are the nodes and the transitions are the edges. An example of a FSM can be seen in Figure \ref{fig:fsm_example}, where a FSM with three states is depicted, in which $s_1$ is the initial state, $s_3$ is the final one and the transitions between states are triggered by receiving a 1 or a 0.

\begin{figure}[!h]
    \centering
    \begin{tikzpicture}[shorten >=1pt,node distance=2cm,on grid,auto]
        \tikzstyle{every state}=[fill={rgb:black,1;white,10}]

        \node[state,initial]   (s_1)                {$s_1$};
        \node[state]           (s_2) [right of=s_1] {$s_2$};
        \node[state,accepting] (s_3) [right of=s_2] {$s_3$};

        \path[->]
        (s_1) edge [loop above] node {0} (   )
        (s_1) edge [bend left]  node {1} (s_2)
        (s_2) edge [bend left]  node {1} (s_3)
        (s_2) edge [loop above] node {0} (   );
    \end{tikzpicture}
    \caption{Example of a FSM}
    \label{fig:fsm_example}
\end{figure}

As described in \cite{BTsInRobotics}, FSMs are a very intuitive control architecture and can be easily implemented, however, they are not very suitable for describing complex systems. FSMs usually do not scale well, as adding more and more states and transitions to the system, makes it harder to understand and modify. Besides that, FSMs have also a problem regarding maintainability, as adding or removing states can result in re-evaluating numerous transitions and internal states, making FSMs prone to human design errors and impractical for automated design purposes.

\subsection{Hierarchical Finite State Machines}

Hierarchical Finite State Machine (HFSM), also referred to as State Charts, is another type of control architecture that is derived from FSMs. HFSMs are based on the concept of superstates, the idea that a state can contain one or more substates. Thus, in HFSMs there is also the concept of \textit{generalized transitions}, which allow transition from one superstate to another instead of countless transitions between substates, reducing the total number of transitions specified in the system. It is also important to note that, each superstate designates one substate as the starting state, which executes whenever a transition to the superstate occurs. An example of HFSM is illustrated in Figure \ref{fig:hfsm_example}.

\begin{figure}
    \centering
    \includegraphics[width=0.75\linewidth]{chapters/background/images/HFSM Example.png}
    \caption{Example of a HFSM controlling a NPC of a combat game. \textit{Patrol}, \textit{Use Rifle}, and \textit{Use Handgun} are superstates. Taken from \cite{BTsInRobotics}}
    \label{fig:hfsm_example}
\end{figure}

As it is possible to observe, HFSMs were developed to address some of the shortcomings of FSMs, trying, for example, to alleviate the problem of the number of transitions in complex systems by the means of using \textit{generalized transitions}. Additionally, this control architecture offers enhanced modularity, allowing tasks to be divided into subtasks, and supports behavior inheritance, enabling a substate to inherit properties from its superstate. However, HFSMs still have some of the same problems as FSMs, such as the difficulty of maintaining and modifying the system, and other problems such as editing the system hierarchy manually.

\section{Behavior Trees}

Behavior Trees (BTs) are a versatile control architecture widely known for their flexibility, modularity, ease of comprehension, and maintainability \cite{BTsInRobotics}. They can be described as a mathematical model that is structured as a directed rooted tree of hierarchical nodes. A BT is composed of two primary types of nodes: the control flow nodes, which are the internal nodes of the tree, and the execution nodes, which are the leaves. By combining various control flow and execution nodes, it is possible to construct a tree that describes an agent's behavior.

As a directed rooted tree, every BT has as its base a root node, which is the one responsible for generating the \textit{tick signal}. This signal is sent from the root throughout the tree until it reaches a leaf node, when this happens, the leaf node is executed and returns a status accordingly, which can be \textit{success}, \textit{failure}, or \textit{running}. The status of the leaf node is propagated back until it returns to the root node, which sends the tick signal all over again throughout the tree, starting a new execution step and continuing the execution of the tree.

The internal nodes of the tree, the so called control flow nodes regulate the execution flow of the tree, determining which nodes are executed and the logic of it. The basic control flow nodes in a BT are the fallback node, sequence node, parallel node, and decorator node. On the other hand, execution nodes define commands that the tree should execute. They can be categorized into action nodes and condition nodes. The characteristics and functionalities of these nodes will be delineated in the subsequent subsection.

\subsection{Types of Nodes}

\subsubsection{Root Node}

As previously explained, the root node is the one responsible for sending the tick signal to the whole tree. The node is usually represented as illustrated in Figure \ref{fig:background_root_node}.

\begin{figure}[!h]
    \centering
    \scalebox{1.0} {
        \begin{forest}
            [\root, controlflow]
        \end{forest}
    }
    \caption{Representation of a root node}
    \label{fig:background_root_node}
\end{figure}

\subsubsection{Fallback Node}

\begin{figure}[!h]
    \centering
    \scalebox{1.0} {
        \begin{forest}
            [\reactivefallback, controlflow]
        \end{forest}
    }
    \caption{Representation of a fallback node}
    \label{fig:background_fallback_node}
\end{figure}

\subsubsection{Sequence Node}

\begin{figure}[!h]
    \centering
    \scalebox{1.0} {
        \begin{forest}
            [\reactivesequence, controlflow]
        \end{forest}
    }
    \caption{Representation of sequence node}
    \label{fig:background_sequence_node}
\end{figure}

\subsubsection{Parallel Node}

\begin{figure}[!h]
    \centering
    \scalebox{1.0} {
        \begin{forest}
            [\parallel, controlflow]
        \end{forest}
    }
    \caption{Representation of a parallel node}
    \label{fig:background_parallel_node}
\end{figure}

\subsubsection{Decorator Nodes}

\begin{figure}[!h]
    \centering
    \scalebox{1.0} {
        \begin{forest}
            [$\mathbf{\delta}$, decorator]
        \end{forest}
    }
    \caption{Representation of a decorator node with policy $\mathbf{\delta}$}
    \label{fig:background_decorator_node}
\end{figure}

\subsubsection{Action Nodes}

\begin{figure}[!h]
    \centering
    \scalebox{1.0} {
        \begin{forest}
            [Action, action, minimum height=7mm, minimum width=10mm]
        \end{forest}
    }
    \caption{Representation of an action node}
    \label{fig:background_action_node}
\end{figure}

\subsubsection{Condition Nodes}

\begin{figure}[!h]
    \centering
    \scalebox{1.0} {
        \begin{forest}
            [Condition, condition, minimum height=7mm, minimum width=10mm]
        \end{forest}
    }
    \caption{Representation of a condition node}
    \label{fig:background_condition_node}
\end{figure}

\subsubsection{Subtree Nodes}

\begin{figure}[!h]
    \centering
    \scalebox{1.0} {
        \begin{forest}
            [{Subtree}, subtree, minimum height=7mm, minimum width=10mm]
        \end{forest}
    }
    \caption{Representation of a subtree node}
    \label{fig:background_subtree_node}
\end{figure}

\subsection{Reactivity}

\begin{figure}[!h]
    \centering
    \begin{subfigure}[b]{.49\linewidth}
        \centering
        \scalebox{1.0} {
            \begin{forest}
                [\fallback, controlflow]
            \end{forest}
        }
        \caption{Non-reactive fallback}
    \end{subfigure}
    \hfill
    \begin{subfigure}[b]{.49\linewidth}
        \centering
        \scalebox{1.0} {
            \begin{forest}
                [\sequence, controlflow]
            \end{forest}
        }
        \caption{Non-reactive sequence}
    \end{subfigure}
    \caption{Representation of non-reactive nodes}
    \label{fig:control_nodes_spec}
\end{figure}

\subsection{BTs vs other Control Architectures}

\section{Blackboard}

\cite{BlackboardDesignPattern}
