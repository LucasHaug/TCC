\def \MOISEp {$\mathcal{M}OISE^+$} 

No mundo acadêmico, as competições universitárias de robótica têm um grande papel no desenvolvimento do cenário da área tanto no quadro nacional quanto no internacional, incentivando a pesquisa e a concepção de sistemas robóticos. No contexto dessas competições, uma das categorias de robôs mais desafiadora e estimulante é a \textit{IEEE Very Small Size Soccer (VSSS)}, que visa desenvolver uma solução completa de engenharia para um time de robôs que jogam futebol autonomamente. A Equipe ThundeRatz de Robótica da Escola Politécnica da USP possui atualmente uma solução para o problema proposto pela categoria, a qual utiliza o \textit{Robotic Operating System (ROS)} para estruturação do sistema e árvores de comportamento para modelagem dos agentes inteligentes que jogam futebol. Essa solução é conhecida como o time ThunderVolt e será o alvo deste trabalho, o qual tem como objetivo aprimorar o time ao desenvolver uma especificação de uma organização de um sistema multiagente com base no modelo \MOISEp, usando árvores de comportamento. Esta organização permitirá a elaboração de uma estratégia de coordenação do time a partir das tecnologias já utilizadas e validando esses aprimoramentos em uma competição acadêmica, a \textit{IRONCup 2023} \footnote{https://events.robocore.net/ironcup-2023/}.
